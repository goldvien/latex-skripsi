%!TEX root = ./template-skripsi.tex
%-------------------------------------------------------------------------------
% 								BAB I
% 							LATAR BELAKANG
%-------------------------------------------------------------------------------

\chapter{LATAR BELAKANG}

\section{Latar Belakang Masalah}
Setiap hari semua manusia di dunia, tanpa terkecuali, membutuhkan makanan untuk dikonsumsi. Untuk mendapatkan makanan yang akan dikonsumsi, manusia dapat memasak sendiri masakannya atau tidak memasak sendiri makanannya. Kendala utama yang dirasakan orang-orang pada umumnya saat memasak adalah hampir 90 persen masyarakat Indonesia disibukkan dengan pekerjaan, sehingga rasa lelah lebih mendera. Hal inilah yang membuat mereka enggan memasak. Semua orang memiliki kemampuan untuk memasak namun terkendala oleh waktu. Menghindari atau enggan membersihkan serta membereskan peralatan masak juga menjadi alasan mengapa orang-orang enggan memasak \cite{poernomo}.

Berdasarkan survei yang dilakukan oleh Fiesta Seafood terhadap perempuan berusia 25 - 45 tahun di lima kota, Jakarta, Depok, Bandung, Surabaya dan Yogyakarta, mengungkapkan bahwa ada tiga alasan utama yang membuat seseorang enggan memasak sendiri di rumah, yaitu kesibukan kerja dan urusan rumah tangga lainnya, tidak tahu bumbu, resep atau cara memasak untuk menghasilkan makanan yang enak, tak memiliki motivasi serta dukungan untuk memasak. 

Riset yang dilakukan \emph{Journal of Nutrition Education and Behaviour} yang diringkas oleh situs tabloidnova.com menemukan, 3 dari 4 ibu bekerja mengaku bahwa mereka tidak punya akses dan waktu untuk membuat makanan yang sehat dan bergizi di rumah \cite{inirani}. Masih menurut survei yang sama, ditemukan pula bahwa 9 dari 10 wanita sebenarnya ingin bisa memasak. Namun, mereka mengakui begitu banyak kendala yang dialami untuk kembali ke dapur. Kendala tersebut, menempati posisi pertama adalah karena tidak memiliki waktu, disusul tidak mempunyai ilmu dan pengetahuan dalam mengolah masakan, serta tidak adanya motivasi dan dukungan yang menyemangati untuk memasak.

Pada era kemajuan teknologi seperti sekarang ini, penulis memilih untuk mengembangkan sebuah aplikasi resep masakan pada \textit{smartphone} berbasis Android untuk membantu masyarakat memasak sendiri makanannya. Terdapat berbagai alasan yang dipertimbangkan oleh penulis mengenai pemilihan \textit{platform} Android ini. Dikutip dari situs Android Authority, terdapat beberapa alasan masyarakat dunia menggunakan ponsel berbasis Android yaitu:
\begin{enumerate}
	\item Harga yang menyesuaikan dengan kebutuhan\\
	Dengan dana yang terbatas sekalipun masyarakat dapat mendapatkan \textit{smartphone} berbasis Android melalui ponsel \emph{low-cost flagship phone}, yaitu ponsel pintar dengan harga yang terjangkau
	\item \emph{Multi-tasking}\\
	Android dapat melakukan berbagai pekerjaan sekaligus dalam satu ponsel. Bahkan Samsung beberapa waktu yang lalu sudah mengeluarkan fitur \emph{multi-window} dimana terdapat beberapa \emph{window} sekaligus dalam satu layar.
	\item Integrasi Google\\
	Perangkat Android terintegrasi dengan berbagai produk Google seperti Gmail, Google Drive, Google Maps, dan banyak lagi. 
	\item Banyaknya aplikasi dan games gratis\\
	Banyaknya aplikasi dan games berkualitas yang dapat dinikmati dan diunduh oleh penggunanya secara gratis melalui Google Play meskipun ada pula aplikasi maupun permainan yang berbayar karena fitur khusus yang ditawarkan kepada penggunanya.
\end{enumerate}

Sebelumnya, penulis menemukan aplikasi serupa yang dikelola oleh PT Kompas Media Nusantara yang dikenal dengan Harian Kompas. Aplikasi tersebut bernama Kompas Recipe. 
\begin{figure}[H]
	\centering
	\includegraphics[width=0.3\textwidth]{gambar/kompas/utama}
	\caption{Halaman Utama Kompas Recipe}
\end{figure} 
Di dalam aplikasi tersebut terdapat kumpulan resep masakan yang dibagi menjadi dua jenis yaitu resep masakan berbayar dan resep masakan gratis. Terdapat banyak resep masakan yang menarik namun berbayar seperti Mie Aceh, Ayam Tangkap, dan Nasi Uduk. Untuk itu, penulis ingin mengembangkan aplikasi dengan resep masakan yang keseluruhannya gratis, sesuai dengan prinsip \textit{Open Source}. Selain itu, Kompas Recipe tidak menyediakan video pembuatan makanan yang terdapat pada resep yang dibaca oleh pengguna. Aplikasi yang akan dibuat oleh penulis akan menyertakan video pembuatan makanan yang terdapat pada resep yang ada di dalam aplikasi tersebut. Penulis akan menamai aplikasi tersebut dengan nama Masak Yuk. Kata-kata tersebut sebenarnya merupakan ajakan bagi para masyarakat untuk memasak sendiri di rumah bersama keluarga, kerabat maupun teman-teman pengguna.


\section{Batasan Masalah}
Karena keterbatasan waktu, dana, tenaga, teori-teori dan supaya penelitian dapat dilakukan secara lebih mendalam maka tidak semua masalah akan diteliti. Berikut merupakan batasan-batasan yang diterapkan oleh peneliti
\begin{itemize}
	\item Peneliti akan membuat aplikasi tersebut dengan menggunakan \emph{software}  IDE (\emph{Integrated Development Environtment}) Android Studio. Peneliti memilih menggunakan Android Studio karena fiturnya yang sudah cukup lengkap dan sangat membantu dalam proses pembuatan aplikasi Android. Selain itu, kelengkapan API (\emph{Application Programming Interface}) dan banyaknya dukungan yang tersedia secara \emph{online} memudahkan peneliti dalam mengembangkan sebuah aplikasi Android.
	\item  Peneliti akan langsung menggunakan \emph{smartphone} berbasis Android dalam proses \emph{debugging} dan \emph{testing}. Versi android yang digunakan adalah Lollipop (Android 5.0) dengan API 21.
	\item Koneksi internet yang cepat, minimal pada kecepatan HSPA
	\item Banyaknya resep masakan yang akan dimasukkan kedalam aplikasi sejumlah 30 resep. Jumlah wishlist maksimal 3 resep.
	\item Uji coba hanya akan dilakukan oleh Ahli yang sudah berpartisipasi dalam proses identifikasi awal.
	\item Elemen-elemen yang akan diujikan pada uji coba ahli adalah konten, tampilan, dan fungsionalitas.
\end{itemize}
\vspace{1cm}
\section{Rumusan Masalah}
Rumusan masalah berdasarkan latar belakang di atas adalah:
\begin{enumerate}
	\item Bagaimanakah bentuk aplikasi resep masakan terintegrasi \emph{food channel} YouTube berbasis Android?
	\item Bagaimana cara mengembangkan aplikasi resep masakan terintegrasi \emph{food channel} YouTube berbasis Android?
	\item Bagaimana cara mengintegrasikan YouTube dengan aplikasi Android?
\end{enumerate}


\section{Tujuan Penelitian}
Tujuan dari penelitian ini adalah: 
\begin{enumerate}
	\item Untuk mengetahui bentuk atau wujud aplikasi resep masakan terintegrasi \emph{food channel} YouTube berbasis Android. 
	\item Untuk memaparkan fitur-fitur yang terdapat pada aplikasi tersebut. Selain itu, peneliti juga ingin menggali dan mengetahui cara mengembangkan aplikasi resep masakan terintegrasi food channel YouTube berbasis Android. 
	\item Untuk mengetahui lebih lanjut bagaimana cara mengintegrasikan YouTube, khususnya \emph{food channel} Youtube, dengan aplikasi Android. 
\end{enumerate}

\section{Manfaat Penelitian}
Penelitian ini memiliki banyak manfaat, yaitu: 
	\begin{enumerate}
		\item Bagi Penulis\\
		Melalui penulisan penelitian ini, penulis dapat memahami cara mengembangkan aplikasi Android secara baik dan benar serta mengetahui cara mengintegrasikan YouTube ke dalam sebuah aplikasi Android
		\item Bagi Program Studi Ilmu Komputer\\
		Penulisan penelitian ini memberikan gambaran bagi seluruh mahasiswa khususnya bagi mahasiswa program studi Ilmu Komputer Universitas Negeri Jakarta tentang bagaimana cara mengembangkan suatu aplikasi Android yang tidak hanya bermanfaat bagi penulis saja tetapi juga bermanfaat untuk masyarakat dari berbagai kalangan, mulai dari anak-anak hingga orang dewasa yang ingin memasak dengan memanfaaatkan teknologi \emph{smartphone} Android sebagai referensi masakan mereka  
		\item Bagi Masyarakat\\
		Penelitian, termasuk aplikasi yag dibuat oleh penulis, dapat dimanfaatkan sebagai sumber referensi utama bagi masyarakat yang ingin memasak dan ingin memanfaatkan \emph{smartphone} sebagai media referensinya.  Menurut penulis, menggunaan \emph{smartphone} sebagai referensi memasak menggantikan buku resep adalah solusi praktis di era digital saat ini karena masyarakat tidak perlu lagi membeli buku atau membawa buku dalam jumlah yang banyak untuk memasak.		
	\end{enumerate}

\section{Jenis Penelitian}
Jenis Penelitian yang dijalani oleh Peneliti berjenis Rekayasa Produk. Jenis penelitian ini mengarahkan penulis kepada pengembangan sebuah sistem informasi berbasis aplikasi ponsel pintar \emph{smartphone}.		
% Baris ini digunakan untuk membantu dalam melakukan sitasi
% Karena diapit dengan comment, maka baris ini akan diabaikan
% oleh compiler LaTeX.
\begin{comment}
\bibliography{daftar-pustaka}
\end{comment}
