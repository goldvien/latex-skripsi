%!TEX root = ./template-skripsi.tex
%-------------------------------------------------------------------------------
%                            	BAB V
%               		KESIMPULAN DAN SARAN
%-------------------------------------------------------------------------------

\chapter{KESIMPULAN DAN SARAN}

\section{Kesimpulan}
	Berdasarkan hasil implementasi dan pengujian konten, tampilan, serta fungsionalitas aplikasi ini, didapat kesimpulan sebagai berikut:

	\begin{enumerate}
		\item Bentuk aplikasi kumpulan resep masakan terintegrasi \textit{food channel} YouTube dapat dilihat pada Gambar \ref{mock-welcome} - Gambar \ref{mock-feedback}. Bentuk spesifik dimana \textit{food channel} YouTube diletakkan pada aplikasi tersebut terdapat pada Gambar \ref{detail_bahan} dan Gambar \ref{detail_cara}.
		
		\item Aplikasi kumpulan resep masakan terintegrasi \textit{food channel} YouTube dikembangkan dengan mengacu pada \textit{System Development Life Cycle} (SDLC) model Spiral dengan tahapan yaitu Identifikasi, Desain, Konstrusi dan Pembangunan, dan yang terakhir adalah Evaluasi.
		
		\item YouTube dapat diintegrasikan dengan aplikasi Android dengan menggunakan YouTube Android Player API. Penggunaan API YouTube tersebut disesuaikan dengan kebutuhannya dalam sebuah antarmuka aplikasi Android. Apabila digunakan pada Activity utuh maka dapat digunakan YouTubePlayerView. Sedangkan jika menggunakan Activity yang terdiri dari satu atau lebih Fragment, maka digunakan YouTubePlayerFragment. 

		\item Berdasarkan hasil uji coba, dapat disimpulkan bahwa \textit{food channel} YouTube berhasil diimplementasikan pada aplikasi yang dikembangkan oleh penulis. Hal tersebut ditunjukkan dengan item penilaian video YouTube dapat dimainkan dengan baik juga menempati peringkat teratas dalam segi tampilan serta video YouTube yang diputar sesuai dengan resepnya, menempati peringkat teratas dalam proses uji coba dari segi konten. Fitur-fitur lainnya turut melengkapi aplikasi resep masakan terintegrasi \textit{food channel} YouTube ini. Dengan kata lain, aplikasi ini berhasil diuji dengan baik dan siap dirilis ke masyarakat luas. 
	\end{enumerate}


\section{Saran}
	Adapun saran-saran penulis untuk penelitian selanjutnya adalah:
	\begin{enumerate}
		\item Menambahkan informasi bagaimana cara memasak yang baik dan benar sesuai dengan standar yang berlaku dalam dunia memasak serta menambah kosakata bahasa asing maupun bahasa daerah untuk memperkaya pengetahuan pengguna aplikasi. 
		
		\item Menambah kapasitas \textit{wishlist} menjadi lebih dari 3.
		
		\item Membuat aplikasi yang dapat menjangkau lebih banyak versi Android, termasuk versi Android sebelum API 21 atau Android 5.0 (Lollipop). 
		
		\item Membuat aplikasi kumpulan resep masakan yang mampu memuat lebih dari satu video pada setiap resepnya.
		
		\item Menggunakan basis data Firebase sebagai pengganti SQL karena Firebase juga merupakan produk Google sama dengan Android. Banyak kemampuan yang mampu dimaksimalkan dengan penggunaan Firebase sebagai basis data dari sebuah aplikasi Android.
		
		\item Jika masih mengembangkan dengan menggunakan CodeIgniter-Based API ataupun API berbasis \textit{web framework} lainnya, penggunaan Retrofit menggantikan Volley sangat disarankan karena kecepatan akses data yang jauh lebih cepat daripada Volley. 

	\end{enumerate}

	
% Baris ini digunakan untuk membantu dalam melakukan sitasi
% Karena diapit dengan comment, maka baris ini akan diabaikan
% oleh compiler LaTeX.
\begin{comment}
\bibliography{daftar-pustaka}
\end{comment}
