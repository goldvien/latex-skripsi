\documentclass{jtetiskripsi}

\usepackage{listings,xcolor}
\usepackage{inconsolata}

\definecolor{dkgreen}{rgb}{0,.6,0}
\definecolor{dkblue}{rgb}{0,0,.6}
\definecolor{dkyellow}{cmyk}{0,0,.8,.3}

\lstset{
	language        = php,
	basicstyle      = \small\ttfamily,
	keywordstyle    = \color{dkblue},
	stringstyle     = \color{red},
	identifierstyle = \color{dkgreen},
	commentstyle    = \color{gray},
	emph            =[1]{php},
	emphstyle       =[1]\color{black},
	emph            =[2]{if,and,or,else},
	emphstyle       =[2]\color{dkyellow}}

\begin{document}
	\begin{lstlisting}
<?php  if ( ! defined('BASEPATH')) exit('No direct script access allowed');

/*
Kode-kode yang terdapat pada file Model ini bertugas untuk
mengelola basis data dan mengeksekusi query basis data
untuk mendapatkan data yang terdapat di dalam
basis data
*/

class Masakyuk_model extends CI_Model {

/*
Inisialisasi hubungan dengan basis data
*/
function __construct(){
parent::__construct();
$this->load->database();
}

/*
Query untuk mendapatkan data dari tabel resep
*/
function get_resep(){
$obj = $this->db->select('*')
->get('tb_resep');
return $obj->result_array();
}

/*
Query untuk mendapatkan data resep yang direkomendasikan dari tabel resep
*/
function get_rekomendasi(){
$obj = $this->db->select('*')
->where('rekomendasi',1)
->get('tb_resep');
return $obj->result_array();
}

/*
Query untuk mendapatkan data resep dengan kategori Makanan Pembuka
dari tabel resep
*/
function get_makanan_pembuka(){
$obj = $this->db->select('*')
->join('tb_kategori','tb_kategori.id_cat = tb_resep.id_cat')
->where('cat',"Makanan Pembuka")
->get('tb_resep');
return $obj->result_array();
}

/*
Query untuk mendapatkan data resep dengan kategori Makanan Utama
dari tabel resep
*/
function get_makanan_utama(){
$obj = $this->db->select('*')
->join('tb_kategori','tb_kategori.id_cat = tb_resep.id_cat')
->where('cat',"Makanan Utama")
->get('tb_resep');
return $obj->result_array();
}

/*
Query untuk mendapatkan data resep dengan kategori Makanan Penutup
dari tabel resep
*/
function get_makanan_penutup(){
$obj = $this->db->select('*')
->join('tb_kategori','tb_kategori.id_cat = tb_resep.id_cat')
->where('cat',"Makanan Penutup")
->get('tb_resep');
return $obj->result_array();
}

/*
Query untuk mendapatkan data bahan dari suatu resep
yang terdapat pada tabel bahan
dengan id resep sebagai parameternya
*/
function get_list_bahan($id_resep){
$obj = $this->db->select('*')
->join('tb_list_bahan','tb_list_bahan.id_bahan = tb_bahan.id_bahan')
->join('tb_satuan','tb_satuan.id_satuan = tb_bahan.id_satuan')
->where('id_resep',$id_resep)
->get('tb_bahan');
return $obj->result_array();
}

/*
Query untuk mendapatkan data cara memasak dari suatu resep
yang terdapat pada tabel cara memasak
dengan id resep sebagai parameternya
*/
function get_list_cara($id_resep){
$obj = $this->db->select('*')
->where('id_resep',$id_resep)
->get('tb_cara_memasak');
return $obj->result_array();
}

/*
Query untuk mendapatkan list resep
berdasarkan kata kunci yang diinginkan
dengan $query sebagai parameternya
yang merupakan kata kunci pencarian
*/
function get_search_bahan($query){
$obj = $this->db->select('*')
->join('tb_kategori','tb_kategori.id_cat = tb_resep.id_cat')
->join('tb_bahan','tb_bahan.id_resep = tb_resep.id_resep')
->join('tb_list_bahan','tb_list_bahan.id_bahan = tb_bahan.id_bahan')
->like('nama_bahan',$query)
->get('tb_resep');
return $obj->result_array();
}

/*
Query untuk mendapatkan data jumlah bahan-bahan
yang diperlukan untuk memasak satu, dua,atau tiga resep sekaligus
yang sudah disimpan dalam wishlist
dengan id resep dari ketiga resep yang tersimpan pada wishlist aplikasi
sebagai parameterya
*/
function get_bahan_accumulation($id_resep,$id_resep_2,$id_resep_3){
if ($id_resep && $id_resep_2 == null && $id_resep_3 == null){
$obj = $this->db->query("SELECT nama_bahan, satuan, SUM(banyaknya)
AS total_banyaknya FROM
(SELECT nama_bahan,banyaknya,satuan FROM tb_bahan
JOIN tb_resep
ON tb_bahan.id_resep = tb_resep.id_resep
JOIN tb_list_bahan
ON tb_list_bahan.id_bahan = tb_bahan.id_bahan
JOIN tb_satuan
ON tb_satuan.id_satuan = tb_bahan.id_satuan
WHERE tb_resep.id_resep = $id_resep
) AS T GROUP BY nama_bahan, satuan");
return $obj->result_array();
} else if ($id_resep && $id_resep_2 && $id_resep_3 == null){
$obj = $this->db->query("SELECT nama_bahan, satuan, SUM(banyaknya)
AS total_banyaknya FROM
(SELECT nama_bahan,banyaknya,satuan FROM tb_bahan
JOIN tb_resep
ON tb_bahan.id_resep = tb_resep.id_resep
JOIN tb_list_bahan
ON tb_list_bahan.id_bahan = tb_bahan.id_bahan
JOIN tb_satuan
ON tb_satuan.id_satuan = tb_bahan.id_satuan
WHERE tb_resep.id_resep = $id_resep
OR tb_resep.id_resep = $id_resep_2
) AS T GROUP BY nama_bahan, satuan");
return $obj->result_array();
} else if ($id_resep && $id_resep_2 && $id_resep_3){
$obj = $this->db->query("SELECT nama_bahan, satuan, SUM(banyaknya)
AS total_banyaknya FROM
(SELECT nama_bahan,banyaknya,satuan FROM tb_bahan
JOIN tb_resep
ON tb_bahan.id_resep = tb_resep.id_resep
JOIN tb_list_bahan
ON tb_list_bahan.id_bahan = tb_bahan.id_bahan
JOIN tb_satuan
ON tb_satuan.id_satuan = tb_bahan.id_satuan
WHERE tb_resep.id_resep = $id_resep
OR tb_resep.id_resep = $id_resep_2
OR tb_resep.id_resep = $id_resep_3
) AS T GROUP BY nama_bahan, satuan");
return $obj->result_array();
}
}

/*
Query untuk mendapatkan data total harga
dari bahan-bahan yang diperlukan untuk memasak satu, dua, atau
tiga resep sekaligus yang sudah disimpan dalam wishlist
dengan id resep dari ketiga resep yang tersimpan pada wishlist aplikasi
sebagai parameterya
*/
function get_harga_accumulation($id_resep,$id_resep_2,$id_resep_3){
if ($id_resep && $id_resep_2 == null && $id_resep_3 == null){
$obj = $this->db->query("SELECT nama_bahan, satuan, harga_per_satuan,
per, SUM(banyaknya) AS total_banyaknya FROM
(SELECT nama_bahan,banyaknya,satuan,harga_per_satuan,per
FROM tb_bahan
JOIN tb_resep
ON tb_bahan.id_resep = tb_resep.id_resep
JOIN tb_list_bahan
ON tb_list_bahan.id_bahan = tb_bahan.id_bahan
JOIN tb_satuan
ON tb_satuan.id_satuan = tb_bahan.id_satuan
JOIN tb_harga
ON tb_bahan.id_bahan = tb_harga.id_bahan
AND tb_bahan.id_satuan = tb_harga.id_satuan
WHERE tb_resep.id_resep = $id_resep
) AS T GROUP BY nama_bahan, satuan");
return $obj->result_array();
} else if ($id_resep && $id_resep_2 && $id_resep_3 == null){
$obj = $this->db->query("SELECT nama_bahan, satuan, harga_per_satuan,
per, SUM(banyaknya) AS total_banyaknya FROM
(SELECT nama_bahan,banyaknya,satuan,harga_per_satuan,per
FROM tb_bahan
JOIN tb_resep
ON tb_bahan.id_resep = tb_resep.id_resep
JOIN tb_list_bahan
ON tb_list_bahan.id_bahan = tb_bahan.id_bahan
JOIN tb_satuan
ON tb_satuan.id_satuan = tb_bahan.id_satuan
JOIN tb_harga
ON tb_bahan.id_bahan = tb_harga.id_bahan
AND tb_bahan.id_satuan = tb_harga.id_satuan
WHERE tb_resep.id_resep = $id_resep
OR tb_resep.id_resep = $id_resep_2
) AS T GROUP BY nama_bahan, satuan");
return $obj->result_array();
} else if ($id_resep && $id_resep_2 && $id_resep_3){
$obj = $this->db->query("SELECT nama_bahan, satuan, harga_per_satuan,
per, SUM(banyaknya) AS total_banyaknya FROM
(SELECT nama_bahan,banyaknya,satuan,harga_per_satuan,per
FROM tb_bahan
JOIN tb_resep
ON tb_bahan.id_resep = tb_resep.id_resep
JOIN tb_list_bahan
ON tb_list_bahan.id_bahan = tb_bahan.id_bahan
JOIN tb_satuan
ON tb_satuan.id_satuan = tb_bahan.id_satuan
JOIN tb_harga
ON tb_bahan.id_bahan = tb_harga.id_bahan
AND tb_bahan.id_satuan = tb_harga.id_satuan
WHERE tb_resep.id_resep = $id_resep
OR tb_resep.id_resep = $id_resep_2
OR tb_resep.id_resep = $id_resep_3
) AS T GROUP BY nama_bahan, satuan");
return $obj->result_array();
}
}

	\end{lstlisting}	
\end{document}